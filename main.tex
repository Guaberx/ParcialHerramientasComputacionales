\documentclass{article}
\usepackage{lmodern}
\usepackage[T1]{fontenc}
\usepackage[spanish]{babel}
\usepackage[utf8x]{inputenc}
\usepackage{mathtools}
\usepackage{graphicx}
\usepackage{listings}




% Styling for the code
\usepackage{listings}
\usepackage{xcolor}

\definecolor{codegreen}{rgb}{0,0.6,0}
\definecolor{codegray}{rgb}{0.5,0.5,0.5}
\definecolor{codepurple}{rgb}{0.58,0,0.82}
\definecolor{backcolour}{rgb}{0.95,0.95,0.92}

\lstdefinestyle{mystyle}{
    backgroundcolor=\color{backcolour},   
    commentstyle=\color{codegreen},
    keywordstyle=\color{magenta},
    numberstyle=\tiny\color{codegray},
    stringstyle=\color{codepurple},
    basicstyle=\ttfamily\footnotesize,
    breakatwhitespace=false,         
    breaklines=true,                 
    captionpos=b,                    
    keepspaces=true,                 
    numbers=left,                    
    numbersep=5pt,                  
    showspaces=false,                
    showstringspaces=false,
    showtabs=false,                  
    tabsize=2
}
\lstset{style=mystyle}

% -----------------------------------


\title{Herramientas Computacionales  - Taller 2}
\author{ Juan Fernando Otoya }
\date{May 9 2022}

\begin{document}

\maketitle
\tableofcontents

\newpage
\section{Introduccion}

Este taller consiste en utilizar LaTex para dar una breve explicacion del codigo realizado para el taller 1.

\section{Arquitectura}

El codigo esta separado en dos partes, una encargada de la logica del negocio, en la que se realizan todos los calculos y operaciones con datos necesarios y una segunda, la interfaz, que tiene como funcion darle al usuario una manera de interactuar con el programa.




\newpage
\section{Codigos}
\subsection{Codigo de la "logica del negocio" de la aplicacion}
\lstinputlisting[language=Python, style=mystyle, label={list:first, caption=Implementacion de el taller 1 - Parqueadero - Logica del negocio}]{Parqueadero.py}

\newpage
\subsection{Interfaz de la aplicacion}
\lstinputlisting[language=Python, style=mystyle, label={list:first, caption=Implementacion de la interfaz parra el taller 1 - Interfaz}]{Interfaz.py}



\end{document}
